\bichapter{绪论}{Introduction}\label{chap:intro}
\let\originalverb\verb
\renewcommand\verb{~\originalverb} % add space before \verb command

研究生学位论文是研究生科学研究工作的全面总结,是描述其研究成果、代表其研究水平的重要学术文献资料,是申请和授予相应学位的基本依据。学位论文撰写是研究生培养过程的基本训练之一,必须按照确定的规范认真执行。指导教师应加强指导,严格把关。
论文撰写应符合国家及各专业部门制定的有关标准,符合汉语语法规范。我校《研究生学位论文撰写规范》参照中华人民共和国国家标准GB7713-87《科学技术报告、学位论文和学术论文的编写格式》的要求制订。
硕士和博士学位论文,除在字数、理论研究的深度及创造性成果等方面的要求不同外,撰写要求基本一致。

\bisection{使用模板}{Use template}
模板使用之初应输入论文信息。模板包含了\verb|ezinfo|宏包,提供键值型接口\verb|bjutset|接收输入的论文信息。每一个键都对应论文的一项信息。模板预设了常用键项,保存在模板一级目录下\verb|ezinfo.cfg|配置文件内。信息输入可参考如下样例,并将其置于主文档的导言区,如:

\begin{center}
\begin{minipage}{.6\textwidth}
\linespread{1}\small
\begin{Verbatim}[frame=single]
\bjutset {%
    clc            = {<中文图书分类号>},
    udc            = {<十进制分类号>},
    schoolcode     = {<学校编号>},
    secretlevel    = {<密级>},
    studentid      = {<学号>},
    % -- 中文信息
    ctitle         = {<中文题目>},
    cauthor        = {<作者>}, 
    cdiscipline    = {<学科>}, 
    cmajor         = {<研究方向>},
    cdegree        = {<学位类型>}, 
    csupervisor    = {<导师>},
    csupervstitle  = {<导师职称>},
    ccollege       = {<学院>},  
    cdate          = {<中文日期>},
    corganization  = {<学位授予单位>},
    % -- 英文信息
    etitle         = {<English Title>},
    edegree        = {<Type of Degree>},
    emajor         = {<Major>},
    eauthor        = {<Author Name>},
    esupervisor    = {<Supervisor Name>},
    ecollege       = {<College Name>},
    edate          = {<Date>}
}
\end{Verbatim}
\end{minipage}
\end{center}

模板还附带了目录宏包\verb|bitoc|,提供了四组命令\verb|\bichapter|、\verb|\bisection|、\verb|\bisubsection|和\verb|\biappendix|用于定义章、节、条以及附录的双语标题。其使用格式为:
\begin{center}\small
\verb|\bichapter{<中文标题>}{<English Title>}|
\end{center}
输入的双语标题会自动列入双语目录中。此外,\verb|bitoc|宏包定义了中、英文摘要、结论章、参考文献、论文成果和致谢环境,无需手动输入中英文标题,相应环境的中英文标题会自动列入双语目录。

\bisection{章节版式}{Format settings}
依据撰写规范,论文正文内容宜分章、节、条、款、项五级,前三级列入目录。各级标题格式示例如下。章标题,三号黑体居中。

\bisection{离离原上草}{li li yuan shang cao}\label{sec:llysc}
节标题,四号黑体顶格,正文另起一行,首行缩进两字符。

\bisubsection{一岁一枯荣}{yi sui yi ku rong}\label{ssec:ysykr}
条标题,小四号黑体顶格,正文另起一行,首行缩进两字符。
\subsubsection{野火烧不尽}\label{sssec:yhsbj}
款、项标题小4号楷体,题序顶格书写,与标题间空一格,下面阐述内容在标题后空一格接排。

\subsubsubsection{春风吹又生}
款标题采用(1)、(2)、……形式的编号,题序空两格书写,以下内容接排。

正文小四号宋体,1.3倍行距。段首缩进两字符。

\bisection{示例}{examples}
\bisubsection{引用示例}{Cross reference examples}
模板中的文献处理使用\verb|gbt7714|宏包,其文献格式排版遵循国标GB/T7714-2015。\verb|gbt7714|宏包的文献引用依赖于\verb|natbib|宏包,且已默认加载,无需额外命令加载\verb|natbib|宏包,引用命令也与\verb|natbib|宏包一致。模板开启了压缩文献编号选项,如遇编号连续的多篇文献会自动使用连字符压缩序号,仅保留首尾序号。

模板使用\verb|cleveref|~宏包实现标签的交叉引用,并已经对引用标签进行了中文适配,适配内容保存在模板一级目录下的\verb|cleveref.cfg|~配置文件内,如有特殊需求可自行修改,配置文件内的设定会被模板自动加载。以下为引用示例。

文献普通引用\cite{同鸣2012},文献叙述引用\citet{Boutsidis2011},多组文献引用\cite{同鸣2012, Lee1999, Tang2013, Ding2006Orthogonal}。

章节引用\cref{chap:intro},章节引用\cref{sec:llysc},章节引用\cref{ssec:ysykr},章节引用\cref{sssec:yhsbj}。

公式引用\cref{eq:long},图引用\cref{fig:flow_chart},表引用\cref{tab:firstone}。

\bisubsection{公式示例}{Equations examples}
公式序号的右侧与右边线顶边排写。

公式较长时最好在等号“=”处转行,如难实现,则可在$ + $、$ - $、$ \times $、$ \div $运算符号处转行,转行时运算符号仅书写于转行式前,不重复书写。

公式中第一次出现的物理量代号应给予注释,注释的转行应与破折号“——\nobreak”后第一个字对齐。破折号占两个字,格式见下例:
\begin{equation}\label{eq:norm}
E=mc^2
\end{equation}
\begin{flushleft}
    \begin{tabularx}{\textwidth}{@{}rr@{~——~}X@{}}
     式中:& $ EEEE $ & 能量; \\
          & $ m $    & 质量; \\
          & $ cc $   & 光速。光速指光在真空中的速率,是一个物理常数。光速指光在
    \end{tabularx}
\end{flushleft}

长公式长公式长公式长公式长公式长公式长公式长公式长公式长公式长公式长公式长公式长公式:
\begin{multline}\label{eq:long}
\frac{1}{2}\Delta(f_{ij}f^{ij})=
2\left(\sum_{i<j}\chi_{ij}(\sigma_i-\sigma_j)^2+f^{ij}\nabla_j\nabla_i(\Delta f)+\right .\\
\left .\nabla_kf_{ij}\nabla^kf^{ij}+f^{ij}f^k\left[2\nabla_iR_{jk}-\nabla_kR_{ij}\right]\vphantom{\sum_{i<j}}\right)
\end{multline}

多行公式,使用\verb|\nonumber|关闭编号:
\begin{align}
&I(X_3;X_4)-I(X_3;X_4\mid{}X_1)-I(X_3;X_4\mid{}X_2)\nonumber\\
=&[I(X_3;X_4)-I(X_3;X_4\mid{}X_1)]-I(X_3;X_4\mid{}\tilde{X}_2)\\
=&I(X_1;X_3;X_4)-I(X_3;X_4\mid{}\tilde{X}_2)
\end{align}

方程组:
\begin{align}
\begin{cases}
\ u_{tt}(x,t)= b(t)\triangle u(x,t-4)&\\
\ \hspace{42pt}- q(x,t)f[u(x,t-3)]+te^{-t}\sin^2 x,  &  t \neq t_k; \\
\ u(x,t_k^+) - u(x,t_k^-) = c_k u(x,t_k), & k=1,2,3\ldots ;\\
\ u_{t}(x,t_k^+) - u_{t}(x,t_k^-) =c_k u_{t}(x,t_k), &
k=1,2,3\ldots\ .
\end{cases}
\end{align}

\bisubsection{图表示例}{Figures and tables examples}
插图应与文字紧密配合,文图相符,内容正确。选图要力求精练。

机械工程图:采用第一角投影法,严格按照GB4457~GB131-83《机械制图》标准规定。

电气图:图形符号、文字符号等应符合\cref{app:codes}~所列有关标准的规定。

流程图:原则上应采用结构化程序并正确运用流程框图。

对无规定符号的图形应采用该行业的常用画法。

每个图均应有图题(由图号和图名组成)。图号按章编排,如第1章第一个插图的图号为“图 1-1”等。图题置于图下,用中、英文两种文字居中书写,中文在上,要求用5号字。有图注或其它说明时应置于图题之上。图名在图号之后空一格排写。引用图应注明出处,在图题右上角加引用文献号 。图中若有分图时,分图题置于分图之下,分图号用a)、b)等表示。图中各部分说明应采用中文(引用的外文图除外)或数字项号,各项文字说明置于图题之上(有分图题者,置于分图题之上)。

Ti\textit{k}Z图形。
\usetikzlibrary{shapes.geometric, arrows}
\tikzstyle{startstop} = [
rectangle,
rounded corners,
minimum width=2cm,
minimum height=1cm,
text centered,
draw=black
]
\tikzstyle{io} = [
trapezium,
trapezium left angle=75,
trapezium right angle=105,
minimum width=1cm,
minimum height=1cm,
text centered,
draw=black
]
\tikzstyle{process} = [
rectangle,
minimum width=2cm,
minimum height=1cm,
text centered,
draw=black
]
\tikzstyle{decision} = [
diamond,
minimum width=2cm,
minimum height=1cm,
text centered,
draw=black]
\tikzstyle{arrow} = [thick, ->, >=stealth]

\begin{figure}[htbp]
	\centering
	\zihao{-5}\begin{tikzpicture}[node distance=1.5cm]
    \node (pic) [startstop] {待测图片};
    \node (bg) [io, below of=pic] {读取背景};
    \node (pair) [process, below of=bg] {匹配特征点对};
    \node (threshold) [decision, below of=pair, yshift=-0.5cm] {多于阈值};
    \node (clear) [decision, right of=threshold, xshift=3cm] {清晰?};
    \node (capture) [process, right of=pair, xshift=3cm, yshift=0.5cm] {重采};
    \node (matrix_p) [process, below of=threshold, yshift=-0.8cm] {透视变换矩阵};
    \node (matrix_a) [process, right of=matrix_p, xshift=3cm] {仿射变换矩阵};
    \node (reg) [process, below of=matrix_p] {图像修正};
    \node (return) [startstop, below of=reg] {配准结果};
     
    %连接具体形状
    \draw [arrow](pic) -- (bg);
    \draw [arrow](bg) -- (pair);
    \draw [arrow](pair) -- (threshold);

    \draw [arrow](threshold) -- node[anchor=south] {否} (clear);

    \draw [arrow](clear) -- node[anchor=west] {否} (capture);
    \draw [arrow](capture) |- (pic);
    \draw [arrow](clear) -- node[anchor=west] {是} (matrix_a);
    \draw [arrow](matrix_a) |- (reg);

    \draw [arrow](threshold) -- node[anchor=east] {是} (matrix_p);
    \draw [arrow](matrix_p) -- (reg);
    \draw [arrow](reg) -- (return);
\end{tikzpicture}

	\bicaption{流程图\label{fig:flow_chart}}{Sample Flow Chart}	
\end{figure}

\cref{fig:test}~为带子图的图形环境。引用子图,如\\fbox{IMAGE}{fig:test:A}~所示。
\begin{figure}[htbp]
    \centering
    \bisubcaptionbox
    {子图图题 A\label{fig:test:A}}
    {Subfigure A}[0.4\textwidth]{\fbox{IMAGE}}%
    \qquad
    \bisubcaptionbox
    {长一点的字图图题 B\label{fig:test:B}}
    {Subfigure long title B}[0.4\textwidth]{\fbox{IMAGE}}%
    \bicaption{中文图题}{English Title}
    \label{fig:test}
\end{figure}

带子图的图形环境,子图仅中文图题。
\begin{figure}[htbp]
    \centering
    \subcaptionbox{子图图题 A}[0.4\textwidth]{\fbox{IMAGE}}%
    \qquad
    \subcaptionbox{长一点的字图图题 B}[0.4\textwidth]{\fbox{IMAGE}}%
    \bicaption{中文图题}{English Title}
\end{figure}

插入表格。表的编排建议采用国际通行的三线表即顶线、底线和栏目线 注意:没有竖线 。其中顶线和底线为0.75pt粗线,栏目线为0.5pt细线,排版三线表必要时可加辅助线,三线表的组成要素包括:表序、表题、项目栏、表体、表注。表头设计应简单明了,尽量不用斜线。表头中可采用化学符号或物理量符号。

每个表格均应有表题(由表序和表名组成)。表序一般按章编排,如第1章第一个插表的序号为 “表 1-1”等。表序与表名之间空一格,表名中不允 许使用标点符号,表名后不加标点。表题置于表上,用中、英文两种文字居中排写,中文在上,要求用5号字。

全表如用同一单位,则将单位符号移至表头右上角,加圆括号。

表中数据应准确无误,书写清楚。数字空缺的格内加横线“—”(占2个数字宽度)。表内文字或数字上、下或左、右相同时,采用通栏处理方式,不允许用$ '' $、同上之类的写法。

表内文字说明,起行空一格、转行顶格、句末不加标点。
\begin{table}[!hpb]
	\centering
	\bicaption[指向一个表格的表目录索引]
	{一个颇为标准的三线表格\footnotemark[1]}
	{ATableExample}
	\label{tab:firstone}
	\begin{tabular}{@{}llr@{}}\toprule
		\multicolumn{2}{c}{Item}\\\cmidrule(r){1-2}
		Animal&Description&Price(\$)\\\midrule
		Gnat&pergram&13.65\\
		&each&0.01\\
		Gnu&stuffed&92.50\\
		Emu&stuffed&33.33\\
		Armadillo&frozen&8.99\\\bottomrule
	\end{tabular}
\end{table}
\footnotetext[1]{这个例子来自\href{http://www.ctan.org/tex-archive/macros/latex/contrib/booktabs/booktabs.pdf}{《Publication quality tables in LATEX》}(booktabs宏包的文档)。这也是一个在表格中使用脚注的例子,请留意与threeparttable实现的效果有何不同。}

下面一个是一个更复杂的表格,用\verb|threeparttable|宏包实现带有脚注的表格,如\\fbox{IMAGE}{tab:footnote}。

\newcolumntype{d}[1]{D{.}{.}{#1}}% or D{.}{,}{#1} or D{.}{\cdot}{#1}
\begin{table}[!htpb]
	\bicaption[出现在表目录的标题]
	{一个带有脚注的表格的例子}
	{A Table with footnotes}
	\label{tab:footnote}
	\centering
	\begin{threeparttable}[b]
		\begin{tabular}{ccd{4}cccc}
			\toprule
			\multirow{2}{6mm}{total}&\multicolumn{2}{c}{20\tnote{1}} & \multicolumn{2}{c}{40} &  \multicolumn{2}{c}{60}\\
			\cmidrule(lr){2-3}\cmidrule(lr){4-5}\cmidrule(lr){6-7}
			&www & \multicolumn{1}{c}{k} & www & k & www & k \\ % 使用说明符 d 的列会自动进入数学模式,使用 \multicolumn 对文字表头做特殊处理
			\midrule
			&$\underset{(2.12)}{4.22}$ & 120.0140\tnote{2} & 333.15 & 0.0411 & 444.99 & 0.1387 \\
			&168.6123 & 10.86 & 255.37 & 0.0353 & 376.14 & 0.1058 \\
			&6.761    & 0.007 & 235.37 & 0.0267 & 348.66 & 0.1010 \\
			\bottomrule
		\end{tabular}
		\begin{tablenotes}
			\item [1] the first note.% or \item [a]
			\item [2] the second note.% or \item [b]
		\end{tablenotes}
	\end{threeparttable}
\end{table}