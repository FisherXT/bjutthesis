\bichapter{绪论}{Introduction}\label{chap:intro}
\cref{chap:intro}绪论正文内容\cite{同鸣2012}正文内容正文内容正文内容正文内容正文内容正文内容正文内容正文内容正文内容正文内容正文内容正文内容正文内容正文内容正文内容正文内容正文内容正文内容正文内容正文内容正文内容正文内容正文内容正文内容正文内容正文内容正文内容正文内容正文内容正文内容正文内容正文内容正文内容正文内容正文内容正文内容正文内容正文内容正文内容正文内容正文内容正文内容正文内容正文内容正文内容正文内容正文内容正文内容正文内容正文内容正文内容正文内容正文内容正文内容正文内容正文内容正文内容正文内容正文内容正文内容正文内容正文内容正文内容。

正文内容正文内容正文内容正文内容正文内容正文内容\cite{Boutsidis2011}正文内容正文内容正文内容正文内容正文内容正文内容正文内容正文内容正文内容正文内容正文内容正文内容正文内容正文内容正文内容正文内容正文内容正文内容正文内容正文内容正文内容正文内容正文内容正文内容正文内容正文内容正文内容正文内容正文内容正文内容正文内容正文内容正文内容正文内容正文内容正文内容正文内容正文内容正文内容正文内容正文内容正文内容正文内容正文内容正文内容正文内容正文内容正文内容正文内容正文内容正文内容正文内容正文内容正文内容正文内容正文内容正文内容正文内容。


\bisection{离离原上草}{li li yuan shang cao}\label{sec:llysc}
\cref{sec:llysc}\citep{同鸣2012}正文内容正文内容\citet{同鸣2012}正文内容正文内容正文内容正文内容\cite{同鸣2012, Lee1999, Tang2013, Ding2006Orthogonal}正文内容正文内容正文内容正文内容正文内容正文内容正文内容正文内容正文内容正文内容正文内容正文内容正文内容正文内容正文内容正文内容正文内容正文内容正文内容正文内容正文内容正文内容正文内容正文内容正文内容正文内容正文内容正文内容正文内容正文内容正文内容正文内容正文内容正文内容正文内容正文内容正文内容正文内容正文内容正文内容正文内容正文内容正文内容正文内容正文内容正文内容正文内容正文内容正文内容正文内容正文内容正文内容正文内容正文内容正文内容正文内容正文内容正文内容正文内容正文内容正文内容正文内容正文内容正文内容正文内容正文内容正文内容正文内容。正文内容正文内容正文内容正文内容正文内容正文内容正文内容正文内容正文内容正文内容正文内容正文内容正文内容正文内容正文内容正文内容正文内容正文内容正文内容正文内容正文内容正文内容正文内容正文内容正文内容正文内容正文内容正文内容正文内容正文内容正文内容正文内容正文内容正文内容正文内容正文内容正文内容正文内容正文内容正文内容正文内容正文内容正文内容正文内容正文内容正文内容正文内容正文内容正文内容正文内容正文内容正文内容正文内容

\begin{multline}
\frac{1}{2}\Delta(f_{ij}f^{ij})=
2\left(\sum_{i<j}\chi_{ij}(\sigma_i-\sigma_j)^2+f^{ij}\nabla_j\nabla_i(\Delta f)+\right .\\
\left .\nabla_kf_{ij}\nabla^kf^{ij}+f^{ij}f^k\left[2\nabla_iR_{jk}-\nabla_kR_{ij}\right]\vphantom{\sum_{i<j}}\right)
\end{multline}

\bisubsection{煮豆燃豆萁}{\pinyin{zhu3dou4ran2dou4qi2}}
\zhlipsum[1-3][name=simp]

参见\cref{eq:none}:
\begin{align}
\label{eq:none}
&I(X_3;X_4)-I(X_3;X_4\mid{}X_1)-I(X_3;X_4\mid{}X_2)\nonumber\\
=&[I(X_3;X_4)-I(X_3;X_4\mid{}X_1)]-I(X_3;X_4\mid{}\tilde{X}_2)\\
=&I(X_1;X_3;X_4)-I(X_3;X_4\mid{}\tilde{X}_2)
\end{align}
\zhlipsum[6-8][name=simp]

\bisubsection{豆在釜中泣}{\pinyin{dou4zai4fu3zhong1qi4}}

\zhlipsum[2-3][name=simp]

\begin{figure}[!htp]
	\centering
	\resizebox{10cm}{!}{\begin{tikzpicture}[node distance=1.5cm]
    \node (pic) [startstop] {待测图片};
    \node (bg) [io, below of=pic] {读取背景};
    \node (pair) [process, below of=bg] {匹配特征点对};
    \node (threshold) [decision, below of=pair, yshift=-0.5cm] {多于阈值};
    \node (clear) [decision, right of=threshold, xshift=3cm] {清晰?};
    \node (capture) [process, right of=pair, xshift=3cm, yshift=0.5cm] {重采};
    \node (matrix_p) [process, below of=threshold, yshift=-0.8cm] {透视变换矩阵};
    \node (matrix_a) [process, right of=matrix_p, xshift=3cm] {仿射变换矩阵};
    \node (reg) [process, below of=matrix_p] {图像修正};
    \node (return) [startstop, below of=reg] {配准结果};
     
    %连接具体形状
    \draw [arrow](pic) -- (bg);
    \draw [arrow](bg) -- (pair);
    \draw [arrow](pair) -- (threshold);

    \draw [arrow](threshold) -- node[anchor=south] {否} (clear);

    \draw [arrow](clear) -- node[anchor=west] {否} (capture);
    \draw [arrow](capture) |- (pic);
    \draw [arrow](clear) -- node[anchor=west] {是} (matrix_a);
    \draw [arrow](matrix_a) |- (reg);

    \draw [arrow](threshold) -- node[anchor=east] {是} (matrix_p);
    \draw [arrow](matrix_p) -- (reg);
    \draw [arrow](reg) -- (return);
\end{tikzpicture}
}
	\bicaption{绘制流程图效果}{Sample Flow Chart}
	\label{fig:flow_chart}
\end{figure}

\bisubsection{本是同根生}{ben shi tong gen sheng}
\zhlipsum[1-3][name=simp]

\subsubsection{款标题}
款、项标题小4号楷体,题序顶格书写,与标题间空一格,下面阐述内容在标题后空一格接排。

\begin{table}[!hpb]
	\centering
	\bicaption[指向一个表格的表目录索引]
	{一个颇为标准的三线表格\footnotemark[1]}
	{ATableExample}
	\label{tab:firstone}
	\begin{tabular}{@{}llr@{}}\toprule
		\multicolumn{2}{c}{Item}\\\cmidrule(r){1-2}
		Animal&Description&Price(\$)\\\midrule
		Gnat&pergram&13.65\\
		&each&0.01\\
		Gnu&stuffed&92.50\\
		Emu&stuffed&33.33\\
		Armadillo&frozen&8.99\\\bottomrule
	\end{tabular}
\end{table}
\footnotetext[1]{这个例子来自\href{http://www.ctan.org/tex-archive/macros/latex/contrib/booktabs/booktabs.pdf}{《Publication quality tables in LATEX》}(booktabs宏包的文档)。这也是一个在表格中使用脚注的例子,请留意与threeparttable实现的效果有何不同。}

下面一个是一个更复杂的表格,用threeparttable实现带有脚注的表格,如表\ref{tab:footnote}。

\newcolumntype{d}[1]{D{.}{.}{#1}}% or D{.}{,}{#1} or D{.}{\cdot}{#1}
\begin{table}[!htpb]
	\bicaption[出现在表目录的标题]
	{一个带有脚注的表格的例子}
	{A Table with footnotes}
	\label{tab:footnote}
	\centering
	\begin{threeparttable}[b]
		\begin{tabular}{ccd{4}cccc}
			\toprule
			\multirow{2}{6mm}{total}&\multicolumn{2}{c}{20\tnote{1}} & \multicolumn{2}{c}{40} &  \multicolumn{2}{c}{60}\\
			\cmidrule(lr){2-3}\cmidrule(lr){4-5}\cmidrule(lr){6-7}
			&www & \multicolumn{1}{c}{k} & www & k & www & k \\ % 使用说明符 d 的列会自动进入数学模式,使用 \multicolumn 对文字表头做特殊处理
			\midrule
			&$\underset{(2.12)}{4.22}$ & 120.0140\tnote{2} & 333.15 & 0.0411 & 444.99 & 0.1387 \\
			&168.6123 & 10.86 & 255.37 & 0.0353 & 376.14 & 0.1058 \\
			&6.761    & 0.007 & 235.37 & 0.0267 & 348.66 & 0.1010 \\
			\bottomrule
		\end{tabular}
		\begin{tablenotes}
			\item [1] the first note.% or \item [a]
			\item [2] the second note.% or \item [b]
		\end{tablenotes}
	\end{threeparttable}
\end{table}

\bisubsection{相煎何太急}{xiang jian he tai ji}
\zhlipsum[1-3][name=simp]

模板中定义了丰富的定理环境
algo(算法),thm(定理),lem(引理),prop(命题),cor(推论),defn(定义),conj(猜想),exmp(例),rem(注),case(情形),
bthm(断言定理),blem(断言引理),bprop(断言命题),bcor(断言推论)。
amsmath还提供了一个proof(证明)的环境。
这里举一个“定理”和“证明”的例子。

%\begin{theorem}[留数定理]
%	\label{thm:res}
	假设$U$是复平面上的一个单连通开子集,$a_1,\ldots,a_n$是复平面上有限个点,$f$是定义在$U\backslash\{a_1,\ldots,a_n\}$上的全纯函数,
	如果$\gamma$是一条把$a_1,\ldots,a_n$包围起来的可求长曲线,但不经过任何一个$a_k$,并且其起点与终点重合,那么:
	
	\begin{equation}
	\label{eq:res}
	\ointop_{\gamma}f(z)\,\mathrm{d}z=2\uppi\mathbf{i}\sum^n_{k=1}\mathrm{I}(\gamma,a_k)\mathrm{Res}(f,a_k)
	\end{equation}
	
	如果$\gamma$是若尔当曲线,那么$\mathrm{I}(\gamma,a_k)=1$,因此:
	
	\begin{equation}
	\label{eq:resthm}
	\ointop_{\gamma}f(z)\,\mathrm{d}z=2\uppi\mathbf{i}\sum^n_{k=1}\mathrm{Res}(f,a_k)
	\end{equation}
	
	%\oint_\gammaf(z)\,dz=2\pii\sum_{k=1}^n\mathrm{Res}(f,a_k).
	
	在这里,$\mathrm{Res}(f,a_k)$表示$f$在点$a_k$的留数,$\mathrm{I}(\gamma,a_k)$表示$\gamma$关于点$a_k$的卷绕数。
	卷绕数是一个整数,它描述了曲线$\gamma$绕过点$a_k$的次数。如果$\gamma$依逆时针方向绕着$a_k$移动,卷绕数就是一个正数,
	如果$\gamma$根本不绕过$a_k$,卷绕数就是零。
	
	定理的证明。
	
%	\begin{proof}
%		首先,由……
%		
%		其次,……
%		
%		所以……
%	\end{proof}
%\end{theorem}

\bisection{一岁一枯荣}{yi sui yi ku rong}
\zhlipsum[6-20][name=zhufu]

\bisection{野火烧不尽}{ye huo shao bu jin}
\zhlipsum[21-40][name=zhufu]

\bisection{春风吹又生}{chun feng chui you sheng}
\zhlipsum[41-70][name=zhufu]