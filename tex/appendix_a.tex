\biappendix{有关数字用法的规定}{you guan shu zi yong fa de gui ding}

按《关于出版物上数字用法的试行规定》(1987年1月1日国家语言文字工作委员会等7个单位公布),除习惯用中文数字表示的以外,一般数字均用阿拉伯数字。
\begin{enumerate}%[label=(\arabic*), left=2\ccwd..\parindent]
	\item 公历的世纪、年代、年、月、日和时刻一律用阿拉伯数字,如20世纪,80年代,4时3刻等。年号要用四位数,如1989年,不能用89年。
	\item 记数与计算(含正负整数、分数、小数、百分比、约数等)一律用阿拉伯数字,如3/4,4.5\%,10个月,500多种等。
	\item 一个数值的书写形式要照顾到上下文。不是出现在一组表示科学计量和具有统计意义数字中的一位数可以用汉字,如一个人,六条意见。星期几一律用汉字,如星期六。邻近两个数字并列连用,表示概数,应该用汉字数字,数字间不用顿号隔开,如三五天,七八十种,四十五六岁,一千七八百元等。
	\item 数字作为词素构成定型的词、词组、惯用语、缩略语等应当使用汉字。如二倍体,三叶虫,第三世界,“七五”规划,相差十万八千里等。
	\item 5位以上的数字,尾数零多的,可改写为以万、亿为单位的数。一般情况下不得以十、百、千、十万、百万、千万、十亿、百亿、千亿作为单位。如 千米可改写为3.45亿千米或 万千米,但不能写为3亿 万千米或3亿4千5百万千米。
	\item 数字的书写不必每格一个数码,一般每两数码占一格,数字间分节不用分位号“,”,凡4位或4位以上的数都从个位起每3位数空半个数码(1/4汉字)。“ ”,不写成“3,000,000”,小数点后的数从小数点起向右按每三位一组分节。一个用阿拉伯数字书写的多位数不能从数字中间转行。
	\item 数量的增加或减少要注意下列用词的概念:
	\begin{enumerate}[1)]
		\item 增加为(或增加到)过去的二倍
		\item 增加(或增加了)二倍
		
	\end{enumerate}
	1),即过去为一,现在为二;2),即过去为一,现在为三;3)超额80\%,即定额为100,现在为180;4)降低到80\%,即过去为100,现在为80;5)降低(或降低了)80\%,即原来为100,现在为20;6)为原数的1/4,即原数为4,现在为1,或原数为1,现在为0.25。
\end{enumerate}


应特别注意在表达数字减小时,不宜用倍数,而应采用分数。如减少为原来的1/2,1/3等。
